\begin{abstract}
The RoboCup is a set of leagues in which teams of football-playing robots compete, a project with the goal of promoting the scientific disciplines of AI and robotics.
This tournament is an initative taken to advance the scientific field of AI and robotics. For this paper, two methods will be explored with the aim to create a working strategy and each method will be tested using its own simulation environment.
The first method utilized Proximal Policy Optimization (PPO) in a multi-agent setup with a centralized critic, similar to MAPPO, to help agents learn synchronized attacking and defending policies in simplified RoboCup SSL scenarios.
The second method incorporated a genetic algorithm (GA) with a behavior tree (BT), where the BT controlled the low-level skills and the GA optimized their parameters.
BT+GA method achieved more stable performance under the tested conditions compared to PPO.
As future work, enhancing the stability of the simulation environments, fine-tuning low-level skills, and investigating other AI approaches are recognized as important steps in the direction of more successful strategies.
\end{abstract}
