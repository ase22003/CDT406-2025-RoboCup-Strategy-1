\section{Introduction}
\label{section:intro}

The RoboCup \cite{RoboCupSSL} is a tournament where different teams compete against each other with soccer playing robots. The RoboCup Federation arranges several types of leagues where every league uses different types of robots in different shapes and sizes. Overall, this tournament aims to advance in the scientific field of mobile robots. 
This project will focus on the Small Size League (SSL), division B in particular. In the SSL division B teams compete in 6 vs 6 matches of two halves where each half is five minutes long with a five-minute pause in between. The robots are constrained to certain physical dimensions according to the rules (the robots need to fit inside a cylinder of 0.18 meters width and 0.15 meters height) and the robots are built by the members of each team. The playing field is 10.4 times 7.4 meters with a playing area of 9 times 6 meters and the game is played with an orange golf ball. The rules of this league are similar to regular soccer but with several modifications. For example the rules include yellow and red cards, freekicks and penalties but also rules like maximum shooting speed and maximum dribbling length. 
The aim of this project is to develop a system that works well in simulation. That will be done by creating an AI system that can coordinate all six robots, handle the ball, score goals and defend against the opponents. In the long term the models we develop could be further developed and used in other works related to both RoboCup and other areas. In this paper we aim to answer the question of how transferable policies trained in simpler, more abstracted simulators like \acronym{VMAS} are to more accurate simulators such as grSim, and at which level of the hierarchal AI model.
